\documentclass{cv}

\begin{document}

\header{Stanisław Nieradko}

\vspace{10pt}

\contacts{
    \contactitem{icons/github.png}{https://github.com/KanarekLife}{https://github.com/KanarekLife}
    \contactitem{icons/mail.png}{mailto:stanislaw@nieradko.com}{stanislaw@nieradko.com}
    \contactitem{icons/phone.png}{tel:+48506257727}{+48 506 257 727}
}

\vspace{10pt}

\main{
    \section{Projects}{
      \project{SSGP}{C\#, ASP.Net Core}{2022}{https://github.com/KanarekLife/SSGP}{
          Currently in my free time I'm working on a platform which is intended to connect ambitious high-school students with companies looking for interns.
      }

      \project{eru}{C\#, ASP.Net Core}{2020}{https://github.com/xxlo-devs/eru}{
          Application developed to send notifications to students of our high school about changes in timetable via website and Facebook Messenger.
          Among other things I built application's front-end web page, XML parser for substitutions,
          prepared underlying database and connected it to application with EF Core.
      }

      \project{JedzenioPlanner}{C\#, ASP.Net Core}{2020}{https://github.com/JedzenioPlanner}{
          Application allowed to generate daily diet based on number of meals and target sum of daily calories.
          Moreover application enabled users to share and find recepis of healthy meals.
          I assembled a team for this project and it was later sent for Hack Heroes 2020 hackathon.
          In backend part of the project, I have designed its domain, set up the authentication with auth0 and built docker image for easier deployments.
      }

      \project{WokLearner}{C\#, ASP.Net Core}{2020}{https://github.com/KanarekLife?tab=repository\&q=WokLearner}{
          Application was built to help with studying for final art's exam from paintings recognition.
          Application included paintings gallery and quiz mode which remembered which paintings were already learnt by the user.
          I have built the application's backend from scratch which included authentication based on Identity Framework and JWT.
      }

      \project{SuperGamblino}{C\#, DSharpPlus}{2020}{https://github.com/SuperGamblino/SuperGamblino}{
          Discord bot that allows to play few casino games directly on any discord server.
          While I was working on a project I created a docker image for easier deployments and cleaned up its architecture by introducing dependency injection and fixing SOLID violations.
          Moreover I enchanced the project's documentation and its user experience.
      }

      \project{proxmox-idmap-helper}{Svelte, Proxmox, TypeScript}{2021}{https://proxmox-idmap-helper.nieradko.com}{
          Small web application, written in Svelte, that helps with configuration of correct LXC.idmap commands.
          I have built it when I was trying to share my Intel IGPU in my homelab server between host and LXC container with Jellyfish.
      }
     }

    \section{Awards}{
      \award{HackHeroes}{2\textsuperscript{nd} place}{2020}{
          I coordinated a team of 4 people in building a JedzenioPlanner project for a HackHeroes 2020 hackathon.
          During the project I assembled the team, divided tasks and co-developed its back-end.
          Project took a 2nd place, despite a record number of competitors from all over Poland.
      }
     }
    \section{Education}{
      \school{XX Liceum Ogólnokształcące im. Zbigniewa Herberta}{2019 - present}{Gdańsk, Poland}
     }
}{
    \section{Skills}{
      \skill{.NET}{
          \skills{Proficient}{
              \begin{itemize}
                  \item C\#
                  \item ASP.NET Core
                  \item Entity Framework Core
                  \item REST API Design
              \end{itemize}
          }
          \skills{Familiar}{
              \begin{itemize}
                  \item SOLID Principles
                  \item OOP Design Patterns
                  \item Clean Code
                  \item Clean Architecture
                  \item DDD
                  \item N-Tier Architecture
                  \item Unit and Integration Testing
              \end{itemize}
          }
      }

      \skill{Front-end}{
          \skills{Familiar}{
              \begin{itemize}
                  \item Semantic HTML
                  \item CSS
                  \item JavaScript
                  \item TypeScript
                  \item Svelte
              \end{itemize}
          }
      }

      \skill{Other}{
          \skills{Languages}{
              \begin{itemize}
                  \item Polish (native)
                  \item English (FCE - C1)
              \end{itemize}
          }
          \skills{Familiar}{
              \begin{itemize}
                  \item Git
                  \item SQL (mostly SQLite)
                  \item NodeJS
                  \item Linux Server Administration
                  \item Ansible
                  \item Docker
                  \item Proxmox
                  \item Python
                  \item PowerShell
                  \item CI / CD (mostly GitHub Actions)
                  \item \LaTeX
              \end{itemize}
          }
      }

      \skill{Soft Skills}{
          \begin{itemize}
              \item Good at teamwork
              \item Creative
              \item Up to date with current trends
              \item Willing to learn
              \item Experienced in working in small groups (2 - 4 people)
          \end{itemize}
      }
     }
    \section{Interests}{
      \begin{itemize}
          \item Homelabs
          \item Gaming
          \item Programming
      \end{itemize}
     }
}

\vspace{15pt}

\rodo{
    Wyrażam zgodę na przetwarzanie moich danych osobowych dla potrzeb niezbędnych do realizacji procesu rekrutacji (zgodnie z ustawą z dnia 10 maja 2018 roku o ochronie danych osobowych (Dz. Ustaw z 2018, poz. 1000) oraz zgodnie z Rozporządzeniem Parlamentu Europejskiego i Rady (UE) 2016/679 z dnia 27 kwietnia 2016 r. w sprawie ochrony osób fizycznych w związku z przetwarzaniem danych osobowych i w sprawie swobodnego przepływu takich danych oraz uchylenia dyrektywy 95/46/WE (RODO).
}

\end{document}